\documentclass{letter}

\usepackage{amsmath,amssymb}
\usepackage[T1]{fontenc}
%\usepackage[utf8x]{inputenc}
\usepackage[utf8]{inputenc}
\usepackage{textcomp,marvosym}

%\newcommand{\sect}[1]{{\textbf{\large #1}}}  % Section.
\newcommand{\TODO}[1]{{\emph{\textbf{TODO:} #1}}}

\DeclareUnicodeCharacter{00B2}{\textsuperscript{2}}
\DeclareUnicodeCharacter{2096}{\textsubscript{k}}
\DeclareUnicodeCharacter{2098}{\textsubscript{m}}
\DeclareUnicodeCharacter{2081}{\textsubscript{1}}
\DeclareUnicodeCharacter{2082}{\textsubscript{2}}
\DeclareUnicodeCharacter{2A09}{$\times$}

\newenvironment{comment}[1]%
  {\vspace{5ex}\par\textsf{#1 ---}\ignorespaces\bfseries}%
  {\par\ignorespacesafterend}
\newenvironment{reply}%
  {\vspace{2ex}\par}%
  {\par\upshape}

\providecommand{\citep}[1]{[#1]}
\providecommand{\citet}[1]{[#1]}

\title{Manuscript MRM--17--17981, authors' response to comments}
\date{March 21, 2019}
\address{Jussi Toivonen, MSc\\
Department of Future Technologies\\
University of Turku\\
email: jupito@iki.fi}

\begin{document}
\begin{letter}{Kathryn L. Penney, Academic Editor \\ PLOS ONE}
\opening{Dear Editors and Reviewers,}

On behalf of my research team, I write in reference to your decision letter
regarding manuscript PONE-D-18-36207. It was rejected with a possibility of
resubmission. We would like to resubmit a revised manuscript. Please find
enclosed our manuscript \emph{Radiomics and machine learning of multisequence
multiparametric prostate MRI:\ towards improved non-invasive prostate cancer
characterization} by Jussi Toivonen, Ileana Montoya~Perez, Parisa Movahedi,
Harri Merisaari, Marko Pesola, Pekka Taimen, Peter J. Boström, Jonne
Pohjankukka, Aida Kiviniemi, Tapio Pahikkala, Hannu J. Aronen, and Ivan Jambor.

We thank the reviewers for their due critisism. We have addressed the weaknesses
in our work that was presented. In the following, we respond point-by-point to
the reviewers' comments.


\begin{comment}{Reviewer 1, Comment 1}
The paper is well written, and the approach is detailed sufficiently. As a
general comment on the recent surge of published radiomics papers, while each
paper identifies significant radiomics features, no effort has been made to
connect any of these features with previously published ones. This manuscript is
no exception. More than 1000 features are extracted from four types of images.
There are long lists of variables for each of the four image types; it is not
clear what will be the impact of these findings for future applications. In
addition, the data are acquired in a single institution with the same magnet and
imaging sequences. Again, it is not clear how applicable the identified features
will be in other circumstances. The models are not validated in an independent
dataset.
\end{comment}

\begin{reply}
It is true that only one data set was used. This is mentioned as drawback in the
Discussion section, line 533. We are planning to analyze the repeatability of
texture features in another study, when we have more data available.
\end{reply}


\begin{comment}{R1.C2}
The authors analyze a very rich and unique dataset. For instance, the DWI data
contains 12 b-values. In my view, there are missed opportunities to address
important questions, such as: is there an added value for mapping T2; which
three b-values result in apparent diffusion coefficient (ADC) with the highest
correlation with Gleason Score.
\end{comment}

\begin{reply}
It is mentioned in the Discussion section, line 556, that adding texture
features from T2 did not improve classifier performance.

The evaluation of b-values was not included in the goals of this study, as it
can be addressed separately.
\end{reply}


\begin{comment}{R1.C3}
Figure 1 is misleading; there is little correspondence between the figure legend
and the content. The Field of View, displayed on T2-w and DWI are different. The
labels of the radiomics features are shifted.
\end{comment}

\begin{reply}
We have now made some corrections and clarifications to the figure elements and
captions.
\end{reply}


\begin{comment}{R1.C4}
Figure 2: Please, list the abbreviations. Why the RP lesions appear differently
on the different sequences?
\end{comment}

\begin{reply}
We have now listed the abbreviations in Figure 2. The lesions look a bit
different in the histology and the sequences, because the different data sets
did not include slices from exactly same locations.
\end{reply}


\begin{comment}{Reviewer 2, Comment 1}
It is not clear why authors extracted imaging features in 2D slices. Please
explain why you did use 3D volume images.
\end{comment}

\begin{reply}
We extracted texture features from 2D slices, because the images were highly
anisotropic (about 5⨉1⨉1 mm³). This is mentioned in Section 2.5, line 185.
Various slices from 3D images were used in order to make use of more data.
\end{reply}


\begin{comment}{R2.C2}
Shape-based features were not used. Do you think that shape-based features are
not relevant to Gleason scores. Please mention this in the paper.
\end{comment}

\begin{reply}
Including shape-based features might be useful for Gleason score
characterization. However, it would require a considerable amount of additional
work. In this manuscript we decided to focus on texture features.

\TODO{Mention how shape features have been used in existing literature.}
\end{reply}


\begin{comment}{R2.C3}
Is there any correlation in imaging features between different image types?
\end{comment}

\begin{reply}
\TODO{How should we test feature correlation between image types?}
\end{reply}


\begin{comment}{R2.C4}
Table 5 shows the improved performance when imaging features from different
image types were combined. The models including all features are not practical
in terms of interpretation and model building. What are the results if 18
statistical features and top 1\% features were combined.
\end{comment}

\begin{reply}
\TODO{Estimate classification performance with a combination of statistical and
top 1\% texture features.}
\end{reply}



\end{letter}
\end{document}
